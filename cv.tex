%%%%%%%%%%%%%%%%%
% This CV created using altacv.cls
% (v1.6.4, 13 Nov 2021) written by LianTze Lim (liantze@gmail.com). Now compiles with pdfLaTeX, XeLaTeX and LuaLaTeX.
%
%% It may be distributed and/or modified under the
%% conditions of the LaTeX Project Public License, either version 1.3
%% of this license or (at your option) any later version.
%% The latest version of this license is in
%%    http://www.latex-project.org/lppl.txt
%% and version 1.3 or later is part of all distributions of LaTeX
%% version 2003/12/01 or later.
%%%%%%%%%%%%%%%%

%% Use the "normalphoto" option instead of "ragged2e,withhyper" if you want a normal photo instead of cropped to a circle
\documentclass[10pt,a4paper,ragged2e,withhyper]{altacv}
%% AltaCV uses the fontawesome5 and packages.
%% See http://texdoc.net/pkg/fontawesome5 for full list of symbols.

% Change the page layout if you need to
\geometry{left=1.25cm,right=1.25cm,top=1.5cm,bottom=1.5cm,columnsep=1.2cm}

% The paracol package lets you typeset columns of text in parallel
\usepackage{paracol}

% Change the font if you want to, depending on whether
% you're using pdflatex or xelatex/lualatex
\ifxetexorluatex
  % If using xelatex or lualatex:
  \setmainfont{Roboto Slab}
  \setsansfont{Lato}
  \renewcommand{\familydefault}{\sfdefault}
\else
  % If using pdflatex:
  \usepackage[rm]{roboto}
  \usepackage[defaultsans]{lato}
  % \usepackage{sourcesanspro}
  \renewcommand{\familydefault}{\sfdefault}
\fi

% Change the colours if you want to
% \definecolor{SlateGrey}{HTML}{2E2E2E}
% \definecolor{LightGrey}{HTML}{666666}
% \definecolor{DarkPastelRed}{HTML}{450808}
% \definecolor{PastelRed}{HTML}{8F0D0D}
% \definecolor{GoldenEarth}{HTML}{E7D192}
\colorlet{name}{black}
\colorlet{tagline}{black}
\colorlet{heading}{black}
\colorlet{headingrule}{black}
\colorlet{subheading}{black}
\colorlet{accent}{black}
\colorlet{emphasis}{black}
\colorlet{body}{black}

% Change some fonts, if necessary
\renewcommand{\namefont}{\Huge\rmfamily\bfseries}
\renewcommand{\personalinfofont}{\footnotesize}
\renewcommand{\cvsectionfont}{\LARGE\rmfamily\bfseries}
\renewcommand{\cvsubsectionfont}{\large\bfseries}


% Change the bullets for itemize and rating marker
% for \cvskill if you want to
\renewcommand{\itemmarker}{{\small\textbullet}}
\renewcommand{\ratingmarker}{\faCircle}

\begin{document}
\name{Ilya Kuznetsov}
\tagline{Senior Java Developer}

%% You can add multiple photos on the left or right
% \photoR{2.8cm}{Globe_High}
% \photoL{2.5cm}{Yacht_High,Suitcase_High}

\personalinfo{
  \location{St. Petersburg, Russia}
%   \mailaddress{Åddrésş, Street, 00000 Cóuntry}
  \email{ikuz2002@gmail.com}
%   \phone{+79117500058}
  \linkedin{kuzznya}
  \github{kuzznya}
  %% You can add your own arbitrary detail with
  %% \printinfo{symbol}{detail}[optional hyperlink prefix]
  % \printinfo{\faPaw}{Hey ho!}[https://example.com/]
  %% Or you can declare your own field with
  %% \NewInfoFiled{fieldname}{symbol}[optional hyperlink prefix] and use it:
  % \NewInfoField{gitlab}{\faGitlab}[https://gitlab.com/]
  % \gitlab{your_id}
  %%
  %% For services and platforms like Mastodon where there isn't a
  %% straightforward relation between the user ID/nickname and the hyperlink,
  %% you can use \printinfo directly e.g.
  % \printinfo{\faMastodon}{@username@instace}[https://instance.url/@username]
}

\makecvheader
%% Depending on your tastes, you may want to make fonts of itemize environments slightly smaller
% \AtBeginEnvironment{itemize}{\small}

%% Set the left/right column width ratio to 6:4.
\columnratio{0.6}

% Start a 2-column paracol. Both the left and right columns will automatically
% break across pages if things get too long.
\begin{paracol}{2}
\cvsection{Experience}

\cvevent{}{Dell Technologies}{}{St. Petersburg, Russia}
\cvevent{Senior Java Developer}{}{Mar 2022 -- Ongoing}{}
\cvevent{Java Developer}{}{May 2021 -- Mar 2022}{}

System security management project
\begin{itemize}
\item Responsible for microservice security management system that manages host patching, password rotation \& other security-related tasks.
\item Participated in designing and implementing new event driven architecture using Kafka. Increased test coverage of multiple microservices.
\end{itemize}
Java 11, Sprint Boot, K8s, RabbitMQ, Kafka, PostgreSQL, ELK

\divider

\cvevent{Java \& microservices mentor}{ITMO University}{Sep 2021 -- Ongoing}{St. Petersburg, Russia}

Practice teacher of the course on microservices and high-performance systems for 4th year students


\divider

\cvevent{Junior Java Developer}{Intellectual Technologies}{Nov 2020 -- May 2021}{St. Petersburg, Russia}

Development of data collecting & analysis system for oil industry. Was responsible for fault-tolerant and efficient data collecting and processing from the data sensors in an oil well.\\
Service-oriented architecture, Spring WebFlux (Project Reactor), InfluxDB

\divider

\cvevent{}{ALPE Consulting}{}{St. Petersburg, Russia}
\cvevent{Junior Java Developer}{}{Dec 2019 -- May 2021}{}
\cvevent{Intern Java Developer}{}{Jul 2019 -- Dec 2019}{}

Developed an extension system for SAP using microservice architecture. The communication with SAP was set up with SOAP protocol. Created both backend and frontend part, used microfrontend approach.\\
Java, Spring Boot, Spring Cloud, Hibernate, Vue.js

%% Switch to the right column. This will now automatically move to the second
%% page if the content is too long.
\switchcolumn

\cvsection{Education}

\cvevent{Bachelor's degree,\\ Software Engineering}{ITMO University}{Sept 2019 -- June 2023}{}

\cvsection{Skills}

\cvevent{Programming Languages}{}{}{}

Proficient:
\cvtag{Java}
\cvtag{Kotlin}
\cvtag{SQL}

Other:
\cvtag{JS}
\cvtag{Python}
\cvtag{Bash}

\divider\smallskip

\cvevent{Frameworks}{}{}{}

\cvtag{Spring (Core, Boot, Cloud, Security, Data)}
\cvtag{Quarkus}
\cvtag{Vue.js}
\cvtag{Hibernate}
\cvtag{Liquibase}

\divider\smallskip

\cvevent{Technologies}{}{}{}

\cvtag{PostgreSQL}
\cvtag{Cassandra}
\cvtag{MongoDB}
\cvtag{Kafka}
\cvtag{RabbitMQ}
\cvtag{InfluxDB}
\cvtag{Redis}

\divider\smallskip

\cvevent{Tools}{}{}{}

\cvtag{Maven}
\cvtag{Gradle}
\cvtag{Docker}
\cvtag{Kubernetes}
\cvtag{Linux}
\cvtag{Git}
\cvtag{GitHub Actions}
\cvtag{Jenkins}


\cvsection{Languages}

\begin{itemize}
\item English - advanced
\item Russian - native
\item French - beginner
\end{itemize}

\cvsection{Projects}

\cvevent{Timepicker}{Helps friends to vote for the best event date}{}{}

\url{https://timepicker.kuzznya.space}\\
\github{kuzznya/timepicker}

\begin{itemize}
\item Full development and deployment to K8s
\item Kotlin, Quarkus, Kafka, Cassandra, Vue.js, WebSockets, Keycloak
\end{itemize}

\divider

\cvevent{Querier}{Simple SQL query builder}{}{}
\github{javaica/querier}

Simple yet useful SQL query builder that simplifies query creation by providing DSL.

%% Yeah I didn't spend too much time making all the
%% spacing consistent... sorry. Use \smallskip, \medskip,
%% \bigskip, \vspace etc to make adjustments.

\end{paracol}

\end{document}
