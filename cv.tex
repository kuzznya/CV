%%%%%%%%%%%%%%%%%
% This CV created using altacv.cls
% (v1.6.4, 13 Nov 2021) written by LianTze Lim (liantze@gmail.com). Now compiles with pdfLaTeX, XeLaTeX and LuaLaTeX.
%
%% It may be distributed and/or modified under the
%% conditions of the LaTeX Project Public License, either version 1.3
%% of this license or (at your option) any later version.
%% The latest version of this license is in
%%    http://www.latex-project.org/lppl.txt
%% and version 1.3 or later is part of all distributions of LaTeX
%% version 2003/12/01 or later.
%%%%%%%%%%%%%%%%

%% Use the "normalphoto" option instead of "ragged2e,withhyper" if you want a normal photo instead of cropped to a circle
\documentclass[10pt,a4paper,ragged2e,withhyper]{altacv}
%% AltaCV uses the fontawesome5 and packages.
%% See http://texdoc.net/pkg/fontawesome5 for full list of symbols.

% Change the page layout if you need to
\geometry{left=1.25cm,right=1.25cm,top=1.5cm,bottom=1.5cm,columnsep=0.8cm}

% The paracol package lets you typeset columns of text in parallel
\usepackage{paracol}

% Change the font if you want to, depending on whether
% you're using pdflatex or xelatex/lualatex
\ifxetexorluatex
  % If using xelatex or lualatex:
  \setmainfont{Roboto Slab}
  \setsansfont{Lato}
  \renewcommand{\familydefault}{\sfdefault}
\else
  % If using pdflatex:
  \usepackage[rm]{roboto}
  \usepackage[defaultsans]{lato}
  % \usepackage{sourcesanspro}
  \renewcommand{\familydefault}{\sfdefault}
\fi

% Change the colours if you want to
% \definecolor{SlateGrey}{HTML}{2E2E2E}
% \definecolor{LightGrey}{HTML}{666666}
% \definecolor{DarkPastelRed}{HTML}{450808}
% \definecolor{PastelRed}{HTML}{8F0D0D}
% \definecolor{GoldenEarth}{HTML}{E7D192}
\colorlet{name}{black}
\colorlet{tagline}{black}
\colorlet{heading}{black}
\colorlet{headingrule}{black}
\colorlet{subheading}{black}
\colorlet{accent}{black}
\colorlet{emphasis}{black}
\colorlet{body}{black}

% Change some fonts, if necessary
\renewcommand{\namefont}{\Huge\rmfamily\bfseries}
\renewcommand{\personalinfofont}{\footnotesize}
\renewcommand{\cvsectionfont}{\LARGE\rmfamily\bfseries}
\renewcommand{\cvsubsectionfont}{\large\bfseries}


% Change the bullets for itemize and rating marker
% for \cvskill if you want to
\renewcommand{\itemmarker}{{\small\textbullet}}
\renewcommand{\ratingmarker}{\faCircle}

\begin{document}
\name{Ilia Kuznetsov}
\tagline{Software Development Engineer}

%% You can add multiple photos on the left or right
% \photoR{2.8cm}{Globe_High}
% \photoL{2.5cm}{Yacht_High,Suitcase_High}

\personalinfo{
  \location{Dublin, Ireland}
%   \mailaddress{Åddrésş, Street, 00000 Cóuntry}
  \email{ikuz2002@gmail.com}
%   \phone{+79117500058}
  \linkedin{kuzznya}
  \github{kuzznya}
  %% You can add your own arbitrary detail with
  %% \printinfo{symbol}{detail}[optional hyperlink prefix]
  % \printinfo{\faPaw}{Hey ho!}[https://example.com/]
  %% Or you can declare your own field with
  %% \NewInfoFiled{fieldname}{symbol}[optional hyperlink prefix] and use it:
  % \NewInfoField{gitlab}{\faGitlab}[https://gitlab.com/]
  % \gitlab{your_id}
  %%
  %% For services and platforms like Mastodon where there isn't a
  %% straightforward relation between the user ID/nickname and the hyperlink,
  %% you can use \printinfo directly e.g.
  % \printinfo{\faMastodon}{@username@instace}[https://instance.url/@username]
}

\makecvheader
%% Depending on your tastes, you may want to make fonts of itemize environments slightly smaller
% \AtBeginEnvironment{itemize}{\small}

%% Set the left/right column width ratio to 6:4.
\columnratio{0.7}

% Start a 2-column paracol. Both the left and right columns will automatically
% break across pages if things get too long.
\begin{paracol}{2}
\cvsection{Experience}

\cvevent{Software Development Engineer II (L5)}{Amazon Web Services}{Nov 2022 -- Ongoing}{Dublin, Ireland}

\textit{AWS ElastiCache – managed Redis, Valkey and Memcached service}

\begin{itemize}
  \item Implemented service update support for ElastiCache Serverless that updates customers' clusters automatically and handles over 15k nodes/day with no disruption for the customer traffic.
  \item Improved vertical scaling and engine upgrade framework, ensuring zero downtime and 99.99\% availability by providing <1s write unavailability during failover.
  \item Led and delivered the cross-team project of a Data Lake-based analysis and reporting for End-of-Life campaigns using PySpark that allowed customers to see affected resources.
  \item Implemented a SQL migration validation framework with a DDL statement parser and provided full CD automation of database migration across 30+ AWS regions, reducing release time by more than 50\%.
\end{itemize}

\divider

%\cvevent{}{Dell Technologies}{}{St. Petersburg, Russia}
\cvevent{Senior Software Development Engineer}{Dell Technologies}{Mar 2022 -- Jul 2022}{St. Petersburg, Russia}
\cvevent{Software Development Engineer}{}{May 2021 -- Mar 2022}{}

\textit{Self-service system security management project}

\begin{itemize}
  \item Led the design and implementation of a new event driven architecture using Kafka, significantly improving reliability of the system.
  \item Led the cross-team project to manage credentials and automated patching of all internal servers across multiple internal clouds over 30+ locations.
  \item Provided self-service web UI and backend for customers to manage and patch their resources.
  \item Improved observability of the system by introducing tracing and structured logging, halving the average incident resolution time.
\end{itemize}

\divider

\cvevent{Junior Java Developer}{Intellectual Technologies}{Nov 2020 -- May 2021}{St. Petersburg, Russia}

\textit{Data collection \& analysis system for oil industry}

\begin{itemize}
  \item Designed and implemented fault-tolerant and efficient data collection and processing from the data sensors, handling over 20k datapoints/sec.
  \item Implemented stream and batch analysis that provided monitoring of the critical parts of the oil well.
  \item Designed a long-term storage solution with intelligent dynamic sampling for collected data (more than 1TB/day).
\end{itemize}

\divider

\cvevent{Junior Java Developer}{ALPE Consulting}{Dec 2019 -- May 2021}{St. Petersburg, Russia}
\cvevent{Intern Java Developer}{}{Jul 2019 -- Dec 2019}{}

%% Switch to the right column. This will now automatically move to the second
%% page if the content is too long.
\switchcolumn

\cvsection{Education}

\cvevent{\normalsize{BSc, Software Engineering} \footnotesize{(GPA 3.3)}}{ITMO University}{Sept 2019 -- June 2023}{}

\cvsection{Skills}

\cvevent{Programming Languages}{}{}{}

\cvtag{Java}
\cvtag{Kotlin}
\cvtag{Go}
\cvtag{SQL}

\cvtag{JavaScript/TypeScript}
\cvtag{Python}
\cvtag{Ruby}
\cvtag{Bash}

\medskip

\cvevent{Technologies}{}{}{}

\cvtag{PostgreSQL}
\cvtag{Cassandra}
\cvtag{MongoDB}
\cvtag{Redis}
\cvtag{Kafka}
\cvtag{RabbitMQ}
\cvtag{InfluxDB}
\cvtag{DynamoDB}

\medskip

\cvevent{Tools}{}{}{}

\cvtag{AWS}
\cvtag{Terraform}
\cvtag{Docker}
\cvtag{Kubernetes}
\cvtag{Linux}
\cvtag{Git}
\cvtag{GitHub Actions}
\cvtag{Jenkins}


\cvsection{Languages}

\begin{itemize}
\item English - advanced
\item Russian - native
\item French - beginner
\end{itemize}

\cvsection{Projects}

\cvevent{Redis Search Replica (Go)}{}{}{}

Redis full text search support implemented as a separate read replica (Go, C, k6)

\github{kuzznya/go-redis-search-replica}

\divider

\cvevent{Letsdeploy}{}{}{}

Project provides an easy way to deploy apps and services (PostgreSQL, Redis, etc.) to Kubernetes using web UI (Go, Vue.js)

\github{kuzznya/letsdeploy}

%% Yeah I didn't spend too much time making all the
%% spacing consistent... sorry. Use \smallskip, \medskip,
%% \bigskip, \vspace etc to make adjustments.

\end{paracol}

\end{document}
